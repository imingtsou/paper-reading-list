% It is an example file showing how to use the 'sigkddExp.cls' 
% LaTeX2e document class file for submissions to sigkdd explorations.
% It is an example which *does* use the .bib file (from which the .bbl file
% is produced).
% REMEMBER HOWEVER: After having produced the .bbl file,
% and prior to final submission,
% you need to 'insert'  your .bbl file into your source .tex file so as to provide
% ONE 'self-contained' source file.
%
% Questions regarding SIGS should be sent to
% Adrienne Griscti ---> griscti@acm.org
%
% Questions/suggestions regarding the guidelines, .tex and .cls files, etc. to
% Gerald Murray ---> murray@acm.org
%

\documentclass{format}
\usepackage{amssymb}
\usepackage{pifont}
\usepackage{pbox}
\usepackage{url}
\usepackage{hyperref}
\usepackage{tabularx}

\newcommand{\cmark}{\ding{51}}
\newcommand{\xmark}{\ding{55}}

\begin{document}
%
% --- Author Metadata here ---
% -- Can be completely blank or contain 'commented' information like this...
%\conferenceinfo{WOODSTOCK}{'97 El Paso, Texas USA} % If you happen to know the conference location etc.
%\CopyrightYear{2001} % Allows a non-default  copyright year  to be 'entered' - IF NEED BE.
%\crdata{0-12345-67-8/90/01}  % Allows non-default copyright data to be 'entered' - IF NEED BE.
% --- End of author Metadata ---

\title{Software Frameworks for Deep Learning at Scale}
%\subtitle{[Extended Abstract]
% You need the command \numberofauthors to handle the "boxing"
% and alignment of the authors under the title, and to add
% a section for authors number 4 through n.
%
% Up to the first three authors are aligned under the title;
% use the \alignauthor commands below to handle those names
% and affiliations. Add names, affiliations, addresses for
% additional authors as the argument to \additionalauthors;
% these will be set for you without further effort on your
% part as the last section in the body of your article BEFORE
% References or any Appendices.

\numberofauthors{2}
%
% You can go ahead and credit authors number 4+ here;
% their names will appear in a section called
% "Additional Authors" just before the Appendices
% (if there are any) or Bibliography (if there
% aren't)

% Put no more than the first THREE authors in the \author command
%%You are free to format the authors in alternate ways if you have more 
%%than three authors.

\author{
%
% The command \alignauthor (no curly braces needed) should
% precede each author name, affiliation/snail-mail address and
% e-mail address. Additionally, tag each line of
% affiliation/address with \affaddr, and tag the
%% e-mail address with \email.
\alignauthor James Fox \\
       \affaddr{Georgia Institute of Technology}\\
       \affaddr{North Avenue}\\
       \affaddr{Atlanta, GA 30332}\\
       \email{foxjas09@gmail.com}
\alignauthor Yiming Zou\\
       \affaddr{Indiana University Bloomington}\\
       \affaddr{107 S. Indiana Avenue}\\
       \affaddr{Bloomington, IN 47405-7000}\\
       \email{yizou@iu.edu}
%\alignauthor Lars Th{\o}rv\"{a}ld\titlenote{This author is the
%one who did all the really hard work.}\\
%       \affaddr{The Th{\o}rv\"{a}ld Group}\\
%       \affaddr{1 Th{\o}rv\"{a}ld Circle}\\
%       \affaddr{Hekla, Iceland}\\
%       \email{larst@affiliation.org}
}
%\additionalauthors{Additional authors: John Smith (The Th{\o}rvald Group,
%email: {\texttt{jsmith@affiliation.org}}) and Julius P.~Kumquat
%(The Kumquat Consortium, email: {\texttt{jpkumquat@consortium.net}}).}
%\date{30 July 1999}
\maketitle

\begin{abstract}
The study and adoption of deep learning methods has led to significant progress in different application domains. As deep learning continues to show promise and its utilization matures, so does the infrastructure and software needed to support it. Various frameworks have been developed in recent years to facilitate both implementation and training of deep learning networks. As deep learning has also evolved to scale across multiple machines, there's a growing need for frameworks that can provide full parallel support. While deep learning frameworks restricted to running on a single machine have been studied and compared, frameworks which support parallel deep learning at scale are relatively less known and well-studied. This paper seeks to bridge that gap by surveying, summarizing, and comparing frameworks which currently support distributed execution, including but not limited to Tensorflow, CNTK, Deeplearning4j, MXNet, H2O, CaffeOnSpark, Theano, and Torch. 
\end{abstract}

\section{Introduction}
Deep learning has been quite successful in improving predictive power in domains such as computer vision and natural language processing. As accuracies continue to  increase, so do the complexity of network architectures and the size of the parameter space. Google's network for unsupervised learning of image features reached a billion parameters \cite{donahue2014decaf}, and was increased to 11 billion parmeters in a separate experiment at Stanford \cite{schmidhuber2015deep}. In the NLP space, Digital Reasoning Systems trained a 160 billion parameter network \cite{trask2015modeling} fairly recently. Handling problems of this size involves scaling up to thousands of cores across many machines, which Google first demonstrated on its distributed DistBelief framework \cite{dean2012large}.

The goal of this paper is to survey the landscape of deep learning frameworks with distributed execution, and compare them across a consistent set of characteristics. These characteristics include release date, core language, user-facing API, computation model, communication model, data parallelism, model parallelism, programming paradigm, fault tolerance, and visualization. This choice of criteria is explained in detail in Section 2. Tensorflow, Deeplearning4j, MXNet, H2O, and CaffeOnSpark were chosen by a combination of factors, including their being open-source, level of documentation, maturity as a product, and adoption by the community. Frameworks currently without distributed support won't be the focus of this discussion, but well-known ones such as Theano, Torch, Caffe (without Spark) have been studied in the past \cite{DBLP:journals/corr/BahrampourRSS15}. 

The rest of the paper is organized as follows: Section 2 explains the set of comparison criteria and summarizes the distributed frameworks according that criteria. Section 3 discusses each distributed framework in detail. Section 4 outlines future directions of work. Section 5 concludes the paper.

%The \textit{proceedings} are the records of a conference.
%ACM seeks to give these conference by-products a uniform,
%high-quality appearance.  To do this, ACM has some rigid
%requirements for the format of the proceedings documents: there
%is a specified format (balanced  double columns), a specified
%set of fonts (Arial or Helvetica and Times Roman) in
%certain specified sizes (for instance, 9 point for body copy),
%a specified live area (18 $\times$ 23.5 cm [7" $\times$ 9.25"]) centered on
%the page, specified size of margins (2.54cm [1"] top and
%bottom and 1.9cm [.75"] left and right; specified column width
%(8.45cm [3.33"]) and gutter size (.083cm [.33"]).
%
%The good news is, with only a handful of manual
%settings\footnote{Two of these, the {\texttt{\char'134 numberofauthors}}
%and {\texttt{\char'134 alignauthor}} commands, you have
%already used; another, {\texttt{\char'134 balancecolumns}}, will
%be used in your very last run of \LaTeX\ to ensure
%balanced column heights on the last page.}, the \LaTeX\ document
%class file handles all of this for you.
%
%The remainder of this document is concerned with showing, in
%the context of an ``actual'' document, the \LaTeX\ commands
%specifically available for denoting the structure of a
%proceedings paper, rather than with giving rigorous descriptions
%or explanations of such commands.

\section{Framework Comparison}
\begin{table*}
\centering
\caption{Open-source Frameworks}
\begin{tabular}{|m{2.6cm}<{\centering}|m{2cm}<{\centering}|m{2cm}<{\centering}|m{2cm}<{\centering}|m{2cm}<{\centering}|m{2cm}<{\centering}|m{2cm}<{\centering}|}
\hline
Platform & Tensorflow & CNTK & Deeplearning4j & MXNet & H2O & CaffeOnSpark\\ \hline\hline
Release Date & 2016 & 2016 & 2015 & 2015 & 2014 & 2016\\
\hline
Core Language & C++  & C++ & Java  & C++ & Java &C++, Scala\\
\hline
API & C++, Python & NDL & Java, Scala & C++, Python, R, Scala, Matlab, Javascript, Go, Julia & Java, R, Python, Scala, Javascript, web-UI & Python, Matlab, Scala\\
\hline
Computation Model & Sync or async & Sync & Sync & Sync or async & Async & Sync\\
\hline
Communication Model & Parameter server & MPI & Iterative MapReduce & Parameter server & Distributed fork-join & MPI Allreduce\\
\hline
Data Parallelism & \cmark & \cmark & \cmark & \cmark& \cmark & \cmark \\
\hline
Model Parallelism & \cmark & N/A & \xmark & \cmark & \xmark & \xmark\\
\hline
Programming Paradigm & Imperative & Imperative & Declarative & Both & Declarative & Declarative\\
\hline
Fault Tolerance & Checkpoint-and-recovery & Checkpoint-and-resume & Checkpoint-and-resume & Checkpoint-and-resume & N/A & N/A\\
\hline
Visualization & Graph-visualization (interactive), training monitoring & Graph visualization (static) & Training monitoring & None & None & Summary Statistics\\
\hline
\end{tabular}
\end{table*}

The relevance of release date, core language, user-facing APIs are self-explanatory. Computation model specifies the nature of data consistency through execution, i.e. whether updates are synchronous or asynchronous. In context of optimization kernels like stochastic gradient descent (SGD), synchronous execution has better convergence guarantees by maintaining consistency or near-consistency with sequential execution. However, asynchronous SGD can exploit more parallelism and train faster, but with less guarantees of convergence speed. Note that frameworks like Tensorflow and MXNet simply leave this as a tradeoff choice for the user. 

The communication model tries to categorize the nature of distributed execution by well-known paradigms. 

Data and model parallelism are the two prevalent opportunities for parallelism in training deep learning networks at the distributed level. In data parallelism, copies of the model, or parameters, are each trained on its own subset of the training data, while updating the same global model. In model parallelism, the model itself is partitioned and trained in parallel. 

Programming paradigm falls into the categories of imperative, declarative, or a mix of both. Conventionally, imperative programming specifies \textit{how} a computation is done, where as declarative programming specifies \textit{what} needs to be done. There is plenty of gray area, but the distinction is made in this paper based on whether the API exposes the user to computation details that require some understanding of the inner math of neural networks (imperative), or whether the abstraction is yet higher (declarative). 

Fault tolerance is included for two reasons. Distributed execution tends to be more failure prone, especially at scale. Furthermore, any failures (not necessarily limited to distributed execution) that interrupt training part-way can be very costly, if all the progress made on the model is simply lost. 

Finally, UI/Visualization is a feature supported to very different degrees across the frameworks studied. The ability to monitor the progress of training and the internal state of networks over time could be useful for debugging or hyperparameter tuning, and could be an interesting direction. Tensorflow and Deeplearning4j both support this kind of visualization. 


\section{Framework Discussion}
\subsection{Tensorflow}
Tensorflow was released by Google Research as open source in November 2015, and included distributed support in 2016. The user-facing APIs are C++ and Python. Programming with Tensorflow leans more imperative. While plenty of abstraction power is expressed in its library, the user will probably also be working with computational primitive wrappers such as matrix operations, element-wise math operators, and looping control. In other words, the user is exposed to some of the internal workings of deep learning networks. Tensorflow treats networks as a directed graph of nodes encapsulating dataflow computation and required dependencies \cite{DBLP:journals/corr/AbadiABBCCCDDDG16}. Each node, or computation, gets mapped to devices (CPUs or GPUs) according to some cost function. This partitions the overall graph into subgraphs, one per device. Cross-device edges are replaced to encode necessary synchronization between device pairs. Distributed execution appears to be a natural extension of this arrangement, except that TCP or Remote Direct Memory Access (RDMA) is used for inter-device communication on separate machines. This approach of mapping subgraphs onto devices also offers potential scalability, because each worker can schedule its own subgraph at runtime instead of relying on a centralized master \cite{DBLP:journals/corr/AbadiABBCCCDDDG16}. Parallelism in Tensorflow can be expressed at several levels, notably both data parallelism and model parallelism. Data parallelism can happen both across and within workers, by training separate batches of data on model replications. Model parallelism is expressed through splitting one model, or its graph, across devices. Model updates can either be synchronous or asynchronous for parameter-optimizing algorithms such as SGD. For fault tolerance, Tensorflow provides checkpointing and recovery of data designated to be persistent, while the overall computation graph is restarted. In terms of other features, TensorBoard is a tool for interactive visualization of a user's network, and also provides time series data on various aspects of the learning network's state during training.

\subsection{Deeplearning4j}
Deeplearning4j is a Java-based deep learning library built and supported by Skymind, a machine learning intelligence company, in 2014. It is an open source product designed for adoptability in industry, where Java is very common. The framework currently interfaces with both Java and Scala, with a Python SDK in-progress. Programming is primarily declarative, involving specifying network hyperparameters and layer information. Deeplearning4j integrates with Hadoop and Spark, or Akka and AWS for processing backends. Distributed execution provides data parallelism through the Iterative MapReduce model \cite{Itera38:online}. Each worker processes its own minibatch of training data, with workers periodically "reducing" (averaging) their parameter data. Deeplearning4j hosts its own Java linear algebra library, which claims 2x or more speedup over Python's Numpy library on large matrix multiplies \cite{ND4J'34:online}. Fault tolerance is not mentioned, although Spark does have built-in fault tolerance mechanisms.

\subsection{MXNet}
MXNet became available in 2015 and was developed in collaboration across several institutions, including CMU, University of Washington, and Microsoft. It currently interfaces with C++, Python, R, Scala, Matlab, Javascript, Go, and Julia. MXNet supports both declarative and declarative expressions; declarative in declaring computation graphs with higher-level abstractions like convolutional layers, and imperative in the ability to direct tensor computation and control flow. Data parallelism is supported by default, and it also seems possible to build with model parallelism. Distributed execution in MXNet generally follows a parameter server model, with parallelism and data consistency managed at two levels: intra-worker and inter-worker \cite{chen2015mxnet}. Devices within a single worker machine maintain synchronous consistency on its parameters. Inter-worker data consistency can either be synchronous, where gradients over all workers are aggregated before proceeding, or asynchronous, where each worker independently updates parameters. This trade-off between performance and convergence speed is left as an option to the user. The actual handling of server updates and requests is pushed down to MXNet's dependency engine, which schedules all operations and performs resource management \cite{chen2015mxnet}. Fault tolerance on MXNet involves checkpoint-and-resume, which must be user-initiated.

\subsection{H2O}
H2O is the open-source product of H2O.ai, a company focused on machine learning solutions. H2O is unique among the other tools discussed here in that it is a complete data processing platform, with its own parallel processing engine (improving on MapReduce) with a general machine learning library. The discussion will be limited to H2O's deep learning component, available since 2014. H2O is Java-based at its core, but also offers API support for Java, R, Python, Scala, Javascript, as well as a web-UI interface \cite{candel2015deep}. Programming for deep learning appears declarative, as model-building involves specifying hyperparameters and high-level layer information. Distributed execution for deep learning follows the characteristics of H2O's processing engine, which is in-memory and can be summarized as a distributed fork-join model (targeting finer-grained parallelism) \cite{Landset2015}. Data parallelism follows the "HogWild!" \cite{recht2011hogwild} approach for parallelizing SGD. Multiple cores handle subsets of training data and update shared parameters asynchronously. Scaling up to multi-node, each node operates in parallel on a copy of the global parameters, while parameters are averaged for a global update, per training iteration \cite{candel2015deep}. There does not seem to be explicit support for model parallelism. Fault tolerance involves user-initiated checkpoint-and-resume. H2O's web tool can be used to build models and manage workflows, as well as some basic summary statistics, e.g. confusion matrix from training and validation.

\subsection{CaffeOnSpark}
CaffeOnSpark is a Spark deep learning package released open-source in early 2016 by Yahoo's Big ML team. It serves as a distributed implementation of Caffe, a framework for convolutional deep learning released by UC Berkeley's computer vision community in 2014. The language interface for CaffeOnSpark is Scala (following Spark), while Caffe itself offers Python and Matlab API. Programming is highly declarative; creating a deep learning network involves specifying layers and hyperparameters, which are compiled down to a configuration file that Caffe then uses. During distributed runtime, Spark launches "executors," each responsible for a partition of HDFS-based training data and trains the data by running multiple Caffe threads mapped to GPUs \cite{Large52:online}. MPI is used to synchronize executor's respective the parameters' gradients in an Allreduce-like fashion, per training batch \cite{Caffe27:online}. In terms of other notable features, Caffe itself hosts a repository of pre-trained models of some popular convolutional networks such as AlexNet or GoogleNet. It also integrates support for data preprocessing, including building LMDB databases from raw data for higher-throughput, concurrent reading. It does not seem that CaffeOnSpark presently offers any fault tolerance other than what comes with Spark.



\input{para/4-scaling}

\section{Conclusion}


%\section{The Body of The Paper}
%Typically, the body of a paper is organized
%into a hierarchical structure, with numbered or unnumbered
%headings for sections, subsections, sub-subsections, and even
%smaller sections.  The command \texttt{{\char'134}section} that
%precedes this paragraph is part of such a
%hierarchy.\footnote{This is the second footnote.  It
%starts a series of three footnotes that add nothing
%informational, but just give an idea of how footnotes work
%and look. It is a wordy one, just so you see
%how a longish one plays out.} \LaTeX\ handles the numbering
%and placement of these headings for you, when you use
%the appropriate heading commands around the titles
%of the headings.  If you want a sub-subsection or
%smaller part to be unnumbered in your output, simply append an
%asterisk to the command name.  Examples of both
%numbered and unnumbered headings will appear throughout the
%balance of this sample document.
%
%Because the entire article is contained in
%the \textbf{document} environment, you can indicate the
%start of a new paragraph with a blank line in your
%input file; that is why this sentence forms a separate paragraph.
%
%\subsection{Type Changes and Special Characters}
%We have already seen several typeface changes in this sample.  You
%can indicate italicized words or phrases in your text with
%the command \texttt{{\char'134}textit}; emboldening with the
%command \texttt{{\char'134}textbf}
%and typewriter-style (for instance, for computer code) with
%\texttt{{\char'134}texttt}.  But remember, you do not
%have to indicate typestyle changes when such changes are
%part of the \textit{structural} elements of your
%article; for instance, the heading of this subsection will
%be in a sans serif\footnote{A third footnote, here.
%Let's make this a rather short one to
%see how it looks.} typeface, but that is handled by the
%document class file. Take care with the use
%of\footnote{A fourth, and last, footnote.}
%the curly braces in typeface changes; they mark
%the beginning and end of
%the text that is to be in the different typeface.
%
%You can use whatever symbols, accented characters, or
%non-English characters you need anywhere in your document;
%you can find a complete list of what is
%available in the \textit{\LaTeX\
%User's Guide}\cite{Lamport:LaTeX}.

%\subsection{Math Equations}
%You may want to display math equations in three distinct styles:
%inline, numbered or non-numbered display.  Each of
%the three are discussed in the next sections.

%\subsubsection{Inline (In-text) Equations}
%A formula that appears in the running text is called an
%inline or in-text formula.  It is produced by the
%\textbf{math} environment, which can be
%invoked with the usual \texttt{{\char'134}begin. . .{\char'134}end}
%construction or with the short form \texttt{\$. . .\$}. You
%can use any of the symbols and structures,
%from $\alpha$ to $\omega$, available in
%\LaTeX\cite{Lamport:LaTeX}; this section will simply show a
%few examples of in-text equations in context. Notice how
%this equation: \begin{math}\lim_{n\rightarrow \infty}x=0\end{math},
%set here in in-line math style, looks slightly different when
%set in display style.  (See next section).

%\subsubsection{Display Equations}
%A numbered display equation -- one set off by vertical space
%from the text and centered horizontally -- is produced
%by the \textbf{equation} environment. An unnumbered display
%equation is produced by the \textbf{displaymath} environment.
%
%Again, in either environment, you can use any of the symbols
%and structures available in \LaTeX; this section will just
%give a couple of examples of display equations in context.
%First, consider the equation, shown as an inline equation above:
%\begin{equation}\lim_{n\rightarrow \infty}x=0\end{equation}
%Notice how it is formatted somewhat differently in
%the \textbf{displaymath}
%environment.  Now, we'll enter an unnumbered equation:
%\begin{displaymath}\sum_{i=0}^{\infty} x + 1\end{displaymath}
%and follow it with another numbered equation:
%\begin{equation}\sum_{i=0}^{\infty}x_i=\int_{0}^{\pi+2} f\end{equation}
%just to demonstrate \LaTeX's able handling of numbering.

%\subsection{Citations}
%Citations to articles \cite{bowman:reasoning, clark:pct, braams:babel, herlihy:methodology},
%conference
%proceedings \cite{clark:pct} or books \cite{salas:calculus, Lamport:LaTeX} listed
%in the Bibliography section of your
%article will occur throughout the text of your article.
%You should use BibTeX to automatically produce this bibliography;
%you simply need to insert one of several citation commands with
%a key of the item cited in the proper location in
%the \texttt{.tex} file \cite{Lamport:LaTeX}.
%The key is a short reference you invent to uniquely
%identify each work; in this sample document, the key is
%the first author's surname and a
%word from the title.  This identifying key is included
%with each item in the \texttt{.bib} file for your article.
%
%The details of the construction of the \texttt{.bib} file
%are beyond the scope of this sample document, but more
%information can be found in the \textit{Author's Guide},
%and exhaustive details in the \textit{\LaTeX\ User's
%Guide}\cite{Lamport:LaTeX}.
%
%So far, this article has shown only the plainest form
%of the citation command, using \texttt{{\char'134}cite}.
%
%You can also use a citation as a noun in a sentence, as
% is done here, and in the \citeN{herlihy:methodology} article;
% use \texttt{{\char'134}citeN} in this case.  You can
% even say, ``As was shown in \citeyearNP{bowman:reasoning}. . .''
% or ``. . . which agrees with \citeANP{braams:babel}...'',
% where the text shows only the year or only the author
% component of the citation; use \texttt{{\char'134}citeyearNP}
% or \texttt{{\char'134}citeANP}, respectively,
% for these.  Most of the various citation commands may
% reference more than one work \cite{herlihy:methodology,bowman:reasoning}.
% A complete list of all citation commands available is
% given in the \textit{Author's Guide}.

%\subsection{Tables}
%Because tables cannot be split across pages, the best
%placement for them is typically the top of the page
%nearest their initial cite.  To
%ensure this proper ``floating'' placement of tables, use the
%environment \textbf{table} to enclose the table's contents and
%the table caption.  The contents of the table itself must go
%in the \textbf{tabular} environment, to
%be aligned properly in rows and columns, with the desired
%horizontal and vertical rules.  Again, detailed instructions
%on \textbf{tabular} material
%is found in the \textit{\LaTeX\ User's Guide}.
%
%Immediately following this sentence is the point at which
%Table 1 is included in the input file; compare the
%placement of the table here with the table in the printed
%dvi output of this document.
%
%\begin{table}
%\centering
%\caption{Frequency of Special Characters}
%\begin{tabular}{|c|c|l|} \hline
%Non-English or Math&Frequency&Comments\\ \hline
%\O & 1 in 1,000& For Swedish names\\ \hline
%$\pi$ & 1 in 5& Common in math\\ \hline
%\$ & 4 in 5 & Used in business\\ \hline
%$\Psi^2_1$ & 1 in 40,000& Unexplained usage\\
%\hline\end{tabular}
%\end{table}
%
%To set a wider table, which takes up the whole width of
%the page's live area, use the environment
%\textbf{table*} to enclose the table's contents and
%the table caption.  As with a single-column table, this wide
%table will "float" to a location deemed more desirable.
%Immediately following this sentence is the point at which
%Table 2 is included in the input file; again, it is
%instructive to compare the placement of the
%table here with the table in the printed dvi
%output of this document.
%
%
%\begin{table*}
%\centering
%\caption{Some Typical Commands}
%\begin{tabular}{|c|c|l|} \hline
%Command&A Number&Comments\\ \hline
%\texttt{{\char'134}alignauthor} & 100& Author alignment\\ \hline
%\texttt{{\char'134}numberofauthors}& 200& Author enumeration\\ \hline
%\texttt{{\char'134}table}& 300 & For tables\\ \hline
%\texttt{{\char'134}table*}& 400& For wider tables\\ \hline\end{tabular}
%\end{table*}
% end the environment with {table*}, NOTE not {table}!

%\subsection{Theorem-like Constructs}
%Other common constructs that may occur in your article are
%the forms for logical constructs like theorems, axioms,
%corollaries and proofs.  There are
%two forms, one produced by the
%command \texttt{{\char'134}newtheorem} and the
%other by the command \texttt{{\char'134}newdef}; perhaps
%the clearest and easiest way to distinguish them is
%to compare the two in the output of this sample document:
%
%This uses the \textbf{theorem} environment, created by
%the \texttt{{\char'134}newtheorem} command:
%\newtheorem{theorem}{Theorem}
%\begin{theorem}
%Let $f$ be continuous on $[a,b]$.  If $G$ is
%an antiderivative for $f$ on $[a,b]$, then
%\begin{displaymath}\int^b_af(t)dt = G(b) - G(a).\end{displaymath}
%\end{theorem}
%
%The other uses the \textbf{definition} environment, created
%by the \texttt{{\char'134}newdef} command:
%\newdef{definition}{Definition}
%\begin{definition}
%If $z$ is irrational, then by $e^z$ we mean the
%unique number which has
%logarithm $z$: \begin{displaymath}{\log_e^z = z}\end{displaymath}
%\end{definition}
%
%Two lists of constructs that use one of these
%forms is given in the
%\textit{Author's  Guidelines}.
% 
%There is one other similar construct environment, which is
%already set up
%for you; i.e. you must \textit{not} use
%a \texttt{{\char'134}newdef} command to
%create it: the \textbf{proof} environment.  Here
%is a example of its use:
%\begin{proof}
%Suppose on the contrary there exists a real number $L$ such that
%\begin{displaymath}
%\lim_{x\rightarrow\infty} \frac{f(x)}{g(x)} = L.
%\end{displaymath}
%Then
%\begin{displaymath}
%l=\lim_{x\rightarrow c} f(x)
%= \lim_{x\rightarrow c}
%\left[ g{x} \cdot \frac{f(x)}{g(x)} \right ]
%= \lim_{x\rightarrow c} g(x) \cdot \lim_{x\rightarrow c}
%\frac{f(x)}{g(x)} = 0\cdot L = 0,
%\end{displaymath}
%which contradicts our assumption that $l\neq 0$.
%\end{proof}
%
%Complete rules about using these environments and using the
%two different creation commands are in the
%\textit{Author's Guide}; please consult it for more
%detailed instructions.  If you need to use another construct,
%not listed therein, which you want to have the same
%formatting as the Theorem
%or the Definition\cite{salas:calculus} shown above,
%use the \texttt{{\char'134}newtheorem} or the
%\texttt{{\char'134}newdef} command,
%respectively, to create it.
%
%\subsection*{A Caveat for the \TeX\ Expert}
%Because you have just been given permission to
%use the \texttt{{\char'134}newdef} command to create a
%new form, you might think you can
%use \TeX's \texttt{{\char'134}def} to create a
%new command: \textit{Please refrain from doing this!}
%Remember that your \LaTeX\ source code is primarily intended
%to create camera-ready copy, but may be converted
%to other forms -- e.g. HTML. If you inadvertently omit
%some or all of the \texttt{{\char'134}def}s recompilation will
%be, to say the least, problematic.
%
%\section{Conclusions}
%This paragraph will end the body of this sample document.
%Remember that you might still have Acknowledgements or
%Appendices; brief samples of these
%follow.  There is still the Bibliography to deal with; and
%we will make a disclaimer about that here: with the exception
%of the reference to the \LaTeX\ book, the citations in
%this paper are to articles which have nothing to
%do with the present subject and are used as
%examples only.
%\end{document}  % This is where a 'short' article might terminate

%ACKNOWLEDGEMENTS are optional
%\section{Acknowledgements}
%This section is optional; it is a location for you
%to acknowledge grants, funding, editing assistance and
%what have you.  In the present case, for example, the
%authors would like to thank Gerald Murray of ACM for
%his help in codifying this \textit{Author's Guide}
%and the \textbf{.cls} and \textbf{.tex} files that it describes.

%
% The following two commands are all you need in the
% initial runs of your .tex file to
% produce the bibliography for the citations in your paper.
\bibliographystyle{abbrv}
\bibliography{references}  % sigproc.bib is the name of the Bibliography in this case
% You must have a proper ".bib" file
%  and remember to run:
% latex bibtex latex latex
% to resolve all references
%
% ACM needs 'a single self-contained file'!
%
%APPENDICES are optional
% SIGKDD: balancing columns messes up the footers: Sunita Sarawagi, Jan 2000.
% \balancecolumns
%\appendix
%Appendix A
%\section{Headings in Appendices}
%The rules about hierarchical headings discussed above for
%the body of the article are different in the appendices.
%In the \textbf{appendix} environment, the command
%\textbf{section} is used to
%indicate the start of each Appendix, with alphabetic order
%designation (i.e. the first is A, the second B, etc.) and
%a title (if you include one).  So, if you need
%hierarchical structure
%\textit{within} an Appendix, start with \textbf{subsection} as the
%highest level. Here is an outline of the body of this
%document in Appendix-appropriate form:
%\subsection{Introduction}
%\subsection{The Body of the Paper}
%\subsubsection{Type Changes and Special Characters}
%\subsubsection{Math Equations}
%\paragraph{Inline (In-text) Equations}
%\paragraph{Display Equations}
%\subsubsection{Citations}
%\subsubsection{Tables}
%\subsubsection{Figures}
%\subsubsection{Theorem-like Constructs}
%\subsubsection*{A Caveat for the \TeX\ Expert}
%\subsection{Conclusions}
%\subsection{Acknowledgements}
%\subsection{Additional Authors}
%This section is inserted by \LaTeX; you do not insert it.
%You just add the names and information in the
%\texttt{{\char'134}additionalauthors} command at the start
%of the document.
%\subsection{References}
%Generated by bibtex from your ~.bib file.  Run latex,
%then bibtex, then latex twice (to resolve references)
%to create the ~.bbl file.  Insert that ~.bbl file into
%the .tex source file and comment out
%the command \texttt{{\char'134}thebibliography}.
% This next section command marks the start of
% Appendix B, and does not continue the present hierarchy
%\section{More Help for the Hardy}
%The acmproc-sp document class file itself is chock-full of succinct
%and helpful comments.  If you consider yourself a moderately
%experienced to expert user of \LaTeX, you may find reading
%it useful but please remember not to change it.

% That's all folks!
\end{document}
