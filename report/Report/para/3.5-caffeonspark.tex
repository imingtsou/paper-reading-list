\subsection{CaffeOnSpark}
CaffeOnSpark is a Spark deep learning package released open-source in early 2016 by Yahoo's Big ML team. It serves as a distributed implementation of Caffe, a framework for convolutional deep learning released by UC Berkeley's computer vision community in 2014. The language interface for CaffeOnSpark is Scala (following Spark), while Caffe itself offers Python and Matlab API. Programming is highly declarative; creating a deep learning network involves specifying layers and hyperparameters, which are compiled down to a configuration file that Caffe then uses. During distributed runtime, Spark launches "executors," each responsible for a partition of HDFS-based training data and trains the data by running multiple Caffe threads mapped to GPUs \cite{Large52:online}. MPI is used to synchronize executor's respective the parameters' gradients in an Allreduce-like fashion, per training batch \cite{Caffe27:online}. In terms of other notable features, Caffe itself hosts a repository of pre-trained models of some popular convolutional networks such as AlexNet or GoogleNet. It also integrates support for data preprocessing, including building LMDB databases from raw data for higher-throughput, concurrent reading. It does not seem that CaffeOnSpark presently offers any fault tolerance other than what comes with Spark.
