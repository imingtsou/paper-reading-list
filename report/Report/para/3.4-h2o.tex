\subsection{H2O}
H2O is the open-source product of H2O.ai, a company focused on machine learning solutions. H2O is unique among the other tools discussed here in that it is a complete data processing platform, with its own parallel processing engine (improving on MapReduce) with a general machine learning library. The discussion will be limited to H2O's deep learning component, available since 2014. H2O is Java-based at its core, but also offers API support for Java, R, Python, Scala, Javascript, as well as a web-UI interface \cite{candel2015deep}. Programming for deep learning appears declarative, as model-building involves specifying hyperparameters and high-level layer information. Distributed execution for deep learning follows the characteristics of H2O's processing engine, which is in-memory and can be summarized as a distributed fork-join model (targeting finer-grained parallelism) \cite{Landset2015}. Data parallelism follows the "HogWild!" \cite{recht2011hogwild} approach for parallelizing SGD. Multiple cores handle subsets of training data and update shared parameters asynchronously. Scaling up to multi-node, each node operates in parallel on a copy of the global parameters, while parameters are averaged for a global update, per training iteration \cite{candel2015deep}. There does not seem to be explicit support for model parallelism. Fault tolerance involves user-initiated checkpoint-and-resume. H2O's web tool can be used to build models and manage workflows, as well as some basic summary statistics, e.g. confusion matrix from training and validation.
